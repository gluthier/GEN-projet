\newcommand\auteur{Tony Clavien, Maxime Guillod, Gabriel Luthier \& Guillaume Milani}
\newcommand\cours{GEN}
\newcommand\ecole{IL --- TIC --- HEIG-VD}
\newcommand\domaine{Mini-Projet}
\newcommand\titre{Frogger}
\newcommand\objectif{\item[\textit{Objectif :}]}
\newcommand\duree{\item[\textit{Dur\'ee :}]}
\newcommand\dateRendu{\item[\textit{Date du rendu :}]}
\newcommand\partageTache{\item[\textit{Partage des t\^aches :}]}
\newcommand\effort{\item[\textit{Effort :}]}

\documentclass[a4paper,11pt]{article}
%
\author{\auteur}
\title{\titre}
\date{\today}

\usepackage{fancyhdr}
\usepackage{graphicx}
\usepackage{amsmath}
\usepackage{listings}
\usepackage{listingsutf8}
\usepackage{color}
\usepackage{enumerate}
\usepackage[utf8]{inputenc}
\usepackage[T1]{fontenc}
\usepackage[frenchb]{babel}
\usepackage{float}
\usepackage{geometry}
\usepackage{amssymb,mathtools,pifont}
\usepackage{enumitem}
\usepackage{xspace}
\usepackage{appendix}
% Liens
\usepackage[hyphens]{url}
\usepackage{hyperref}
\geometry{verbose,tmargin=2.5cm,bmargin=2cm,lmargin=1.8cm,rmargin=1.8cm}
\selectlanguage{frenchb}
\frenchbsetup{StandardLists=true}
\DeclareGraphicsExtensions{.pdf,.png,.jpg}
\setlength\parindent{0pt}
\setlength{\parskip}{0.7em}

\usepackage{color}
\definecolor{light-gray}{gray}{0.95}
\usepackage{listings}
\lstset{
	breaklines=true,
	breakatwhitespace=true,
	backgroundcolor=\color{light-gray}
}

\usepackage{tabularx} % for 'tabularx' environment

% headers & footers
\pagestyle{fancy}

\lhead{\domaine}
\rhead{\titre\space\includegraphics[scale=0.03]{../Logo/logo.jpg}}

\renewcommand{\footrulewidth}{0.4pt}% default is 0pt
\lfoot{\auteur}
\cfoot{}
\rfoot{\thepage}

%%%%%%%%%%%%%%%%%%%%%%%%%%%%%%%%%%%%%%%
%%%%%%% BEGIN DOCUMENT
%%%%%%%%%%%%%%%%%%%%%%%%%%%%%%%%%%%%%%%

\begin{document}
\clearpage\maketitle
\thispagestyle{empty}

	\maketitle
	\begin{figure}[h!]
		\centering
		\includegraphics[scale=0.7]{../Logo/logo.jpg}
	\end{figure}
	\newpage

	% % Entete première page
	% \thispagestyle{empty}
	% %
	% \noindent \cours \hfill \ecole{} \newline
	% \noindent \auteur \hfill \today \newline
	% \hrule
	% \vspace{7mm}
	% \noindent {\large \bf \domaine } \hfill \titre {\large \bf }\\[3mm]
	% \hrule

	\tableofcontents
	\listoffigures

	% On a pas de tableau
	% \listoftables

	\newpage
	
	\section{Introduction}
	
	
	\section{Analayse}
	
	\subsection{Règles du jeu}
	
	\subsection{Partage des responsabilités entre le serveur et le client}
	
	\subsection{Diagramme d'activité général}
	
	\subsection{Cas d'utilisation}
	\subsubsection{Diagramme général de contexte}
	\subsubsection{Description des acteurs}
	\subsubsection{Scénario principal}
	\subsubsection{Scénarios alternatifs significatifs}
	
	\subsection{Modèle de domaine}
	\subsubsection{Modèle de domaine pour le client}
	\subsubsection{Modèle de domaine pour le serveur}
	
	\subsection{Base de données}
	\subsubsection{Modèle conceptuel (entité-associations)}
	
	
	\section{Conception du projet}
	
	\subsection{Protocole d'échange entre le client et le serveur}
	
	\subsection{Diagrammes de classes du serveur et du client}
	
	\subsection{Modèle conceptuel \& relationnel de la base de données}
	
	
	\section{Implémentation du projet}
	
	\subsection{Technologies, langages, bibliothèques utilisés}
	
	\subsection{Technologies "originales" (autres que les technologies étudiées à l'école)}
	\subsubsection{Descriptif}
	\subsubsection{Avantages}
	\subsubsection{Limitations}
	\subsubsection{Remarques personnelles}
	
	\subsection{Problèmes éventuels rencontrés et solutions apportées}
	
	
	\section{Gestion du projet}
	
	\subsection{Rôle des participants au sein du groupe de développement}
	
	\subsection{Plan d'itérations initial}
	Pour chaque itération:
	\begin{enumerate}
		\item Objectifs (exprimés en termes de case d'utilisation)
		\item Durée, dates
		\item Qui fait quoi
		\item Charge estimée en heures
	\end{enumerate}
	
	\subsection{Suivi du projet}
	Itérations par itérations (bilan, problèmes rencontrés, replanifications, ...), synthèse Trello (sur une page environ)
	
	\subsection{Stratégie de tests}
	\begin{enumerate}
		\item Effectués quand, par qui
		\item Outils utilisés?
		\item Utilisation de JUnit pour au moins une classe conséquente
		\item Résultats des tests
	\end{enumerate}

	\subsection{Stratégie d'intégration du code de chaque participant (GIT)}


	\section{État des lieux}
	
	\subsection{Ce qui fonctionne (résultats des tests)}
	
	\subsection{Ce qu'il resterait à développer (en proposant une planification)}
	
	
	\section{Auto-critique}
	\begin{enumerate}
		\item Relativement à votre solution technique, votre gestion de projet, votre plan d'itération
		\item Ce que vous auriez pu améliorer et comment
	\end{enumerate}


	\section{Conclusion}
	
	
	\appendix
	\section{Manuel d'utilisation} \label{app:manuelUtil}
	\begin{enumerate}
		\item Installation
		\item Utilisation (avec copies d'écran)
	\end{enumerate}


\end{document}
