\newcommand\auteur{Tony Clavien, Maxime Guillod, Gabriel Luthier \& Guillaume Milani}
\newcommand\cours{GEN}
\newcommand\ecole{IL --- TIC --- HEIG-VD}
\newcommand\domaine{Mini-Projet}
\newcommand\titre{Frogger}

\documentclass[a4paper,11pt]{article}
%
\author{\auteur}
\title{\titre}
\date{\today}

\usepackage{fancyhdr}
\usepackage{graphicx}
\usepackage{amsmath}
\usepackage{listings}
\usepackage{listingsutf8}
\usepackage{color}
\usepackage{enumerate}
\usepackage[utf8]{inputenc}
\usepackage[frenchb]{babel}
\usepackage{float}
\usepackage{geometry}
\usepackage{amssymb,mathtools,pifont}
\usepackage{enumitem}
\usepackage{xspace}
\geometry{verbose,tmargin=2.5cm,bmargin=2cm,lmargin=2cm,rmargin=2.5cm}
\selectlanguage{frenchb}
\frenchbsetup{StandardLists=true}
\DeclareGraphicsExtensions{.pdf,.png,.jpg}
\setlength\parindent{0pt}
\setlength{\parskip}{1em}

% headers & footers
\pagestyle{fancy}

\lhead{\domaine}
\rhead{\titre}

\lfoot{\auteur}
\cfoot{}
\rfoot{\thepage}

%%%%%%%%%%%%%%%%%%%%%%%%%%%%%%%%%%%%%%%
%%%%%%% BEGIN DOCUMENT
%%%%%%%%%%%%%%%%%%%%%%%%%%%%%%%%%%%%%%%

\begin{document}
\clearpage\maketitle
\thispagestyle{empty}

	\maketitle
	\begin{figure}[h!]
		\centering
		\includegraphics[scale=0.7]{../Logo/logo.jpg}
	\end{figure}
	\newpage

	% % Entete première page
	% \thispagestyle{empty}
	% %
	% \noindent \cours \hfill \ecole{} \newline
	% \noindent \auteur \hfill \today \newline
	% \hrule
	% \vspace{7mm}
	% \noindent {\large \bf \domaine } \hfill \titre {\large \bf }\\[3mm]
	% \hrule

	\tableofcontents
	\listoffigures
	\listoftables

	\newpage

	\section{Fonctionnement général}
	\subsection{Objectifs de base}
	Le but de ce projet est de réaliser une application client-serveur dans le cadre du cours GEN. Nous avons décidé de proposer un jeu de type « Frogger » en version valaisanne. \par

	Le but du jeu sera de faire descendre une piste de ski à un Valaisan. Il devra éviter les skieurs genevois la traversant la piste à vive allure.


	\begin{figure}[h!]
		\centering
		\fbox{\includegraphics[scale=0.7]{../Screenshots/demo.png}}
			\caption{Maquette d'une partie}
			\label{maquette}
	\end{figure}

	Les zones rouges sont les zones à disposition du joueur 2 pour placer les obstacles, les sapins sont des obstacles fixes. Le joueur 1 gagne s'il atteint la zone en vert sans heurter d'obstacle.

	\subsection{Utilisation de l'applicatif}
	Le joueur 1 contrôle son personnage à l'aide des flèches du clavier ($\leftarrow \downarrow \rightarrow$) et tente d'éviter les obstacles. Le joueur 2 place les obstacles dans les colonnes, ceci fait que l'obstacle se met en mouvement dans la colonne. Une bouteille de Fendant\texttrademark se vide au fur et à mesure que le joueur 2 place des obstacles. Il doit ensuite attendre qu'elle se remplisse à nouveau pour placer de nouveaux obstacles.\par

	L'administrateur peut régler les paramètres du jeu (« skin » du jeu, vitesse des obstacles, vitesse de rechargement de la bouteille de Fendant\texttrademark). L'administrateur est un rôle hérité de celui de joueur (un administrateur est donc avant tout un joueur). Ce droit est stocké dans la base de données.

	\subsection{Règles du jeu}
	Pour le joueur 1: le but est d'arriver en bas de la piste sans avoir heurté d'obstacle. \\
	Pour le joueur 2: le but est que le joueur 1 heurte un obstacle.

	\subsection{Contraintes}
	Le joueur 1 ne peut pas revenir en arrière, une fois descendu une portion de piste il peut s'arrêter, changer de «colonne», continuer à descendre. \par

	Le joueur 2 peut envoyer des obstacles pour autant que la bouteille de Fendant\texttrademark ne soit pas vide.

	\subsection{Priorités de développement}

	\begin{enumerate}
		\item Première version du jeu «standalone» avec un seul joueur qui prend le rôle du joueur 1 (descendre la piste). Les obstacles sont envoyés aléatoirement. (En parallèle, développement du serveur et de l'API).
		\item Ajout du deuxième joueur en local (les deux jouent sur la même machine avec des touches différentes du clavier).
		\item Les joueurs jouent chacun sur leur machine et se connectent à un serveur distant centralisant la partie. Lorsqu'un joueur se connecte, il voit la liste des parties crées (soit en attente d'un joueur, soit en cours). Il peut soit rejoindre une partie en attente d'un joueur soit créer une nouvelle partie.
		\item L'administrateur peut configurer le serveur (paramètres des parties).
	\end{enumerate}

	\subsection{Base de données}
	La base de données stocke et partage les informations suivantes entre les joueurs et l'administrateur :
	\begin{itemize}
		\item Informations des joueurs (nom d'utilisateur, mot de passe, rôle, points totaux gagnés)
		\item Informations des parties (joueurs, points remportés par chacun)
		\item Paramètres de l'application (ce qui peut être configuré par l'administrateur)
		\item Logs de l'application
	\end{itemize}


	\section{Partage des responsabilité serveur --- client}
	
	\section{Rôle des participants}
	
		Joueurs : 2 personnes par jeu, l'un ayant pour but de réussir le jeu, l'autre l'en empêcher. \\
		Administrateur : Peut changer les règles du jeu, et s'occuper des données (nettoyer, modifier des joueurs, etc...)

	\section{Cas d'utilisation}
		\subsection{Acteurs}
			joueurs, dans une partie nommés : joueur 1 et joueur 2 \\
			administrateur
		\subsection{Scénario principal (succès)}
		\begin{enumerate}
			\item l'un des deux joueurs se connecte au lobby et crée une partie.
			\item le second joueur rejoint la partie libre du joueur 1 dans le lobby.
			\item Dans la préparation de partie, les joueurs choisissent leurs objectifs (traverser la piste, ou empêcher l'adversaire de le faire)
			\item Lorsque les 2 joueurs sont prêts le jeu commence.
			\begin{enumerate}
				\item "L'attaquant" tente de traverser la piste en déplaçant son personnage.
				\item "Le défenseur" tente de l'en empêcher en envoyant des obstacles.
				\item La partie se termine si l'attaquant n'a plus de vie ou si il a atteint l'autre côté. 
			\end{enumerate}
			\item À la fin de la partie, soit les rôles s'inversent et on recommande à l'étape 4 soit les joueurs quittent la partie, la partie est donc interrompus et les joueurs retournent au lobby.
		\end{enumerate}
			
		\subsection{Extensions (ou scénarios alternatifs)}
		Pour permettre la récupération des parties de manière correct, s'assurer que tous les états et les	événements sensibles du système peuvent être récupérés à n'importe quelle étape du scénario. \\
		
		Lors d'une partie, si l'un des deux joueurs est déconnecté, l'adversaire gagne automatiquement le jeu et est renvoyé au lobby.
		
		\subsection{Scénario d'administration}
		\begin{enumerate}
			\item l'administrateur se connecte à l'interface.
			\item Il modifie l'une des options à sa disposition :
			\begin{enumerate}
				\item Les paramètres du jeu.
				\item Les données d'un joueur.
			\end{enumerate}
			\item puis il peut se déconnecte ou recommencer à l'étape 2.
		\end{enumerate}
	

	\newpage
	\section{Protocole d'échange entre le client et le serveur}
		\subsection{Communication}
			Nous allons utiliser JSON\footnote{JSON : JavaScript Object Notation} comme protocole d'échange entre le client et le serveur. \\
			En effet, de part la structure de son contenu, l'échange d'information entre les deux intervenants est très facilement sérialisable/désérialisable, sans compter sur le fait que le contenu est lisible par un individu (contrairement à un contenu binaire).
		\subsection{Données échangées}

	\section{Ébauche du modèle de domaine}

	\section{Base de donnée}

	\newpage
	\section{Plan d'itération}
		\begin{enumerate}
			\item Création de l'interface de base
			\item Ajout du 1er joueur avec les obstacles
			\item Ajout du 2ème joueur en local
			\item Mise en place des logiques de jeu
				\subitem Conditions de départ
				\subitem Conditions de victoire
				\subitem Barre de recharge pour la pose d'obstacle
			\item Ajout de la communication avec le serveur
			\item Application administrateur
			\item Ajout du Lobby
			\item Finalisation 
		\end{enumerate}

\end{document}
